\documentclass{article}

%preamble
\author{Wei Zou (wz299), Jiahe Xu (jx266)}
\title{ORIE 4741 Project Proposal\\ 
\large Zillow Prize: Zillow's Home Value Prediction (Zestimate)
}
\date{Sep 22,2017}

\begin{document}

\maketitle % prints the preamble 

\section{Background}
Zillow?s Zestimate home valuation has influenced the U.S. real estate industry since first released 11 years ago. A house is always the most important purchase a person makes in his lifetime. In this case, it is incredibly important to ensure homeowners have a trusted way to monitor this asset. The Zestimate was created to give consumers as much information as possible about homes and the housing market, marking the first-time consumers had access to this type of home value information at no cost.

\section{Object} 
In our project, we are going to build a model to improve Zestimate residual error using linear regression, logistic regression and random forest methods. Specifically, it is to predict the log-error between their Zestimate and the actual sale price, given all the features of a home. The log error is defined as:
\begin{center}
 logerror=log(Zestimate)-log(SalePrice)
 \end{center}


\section{Partial Data}
\begin{center}
  \begin{tabular}{ | l | l | l |}
    \hline
    'airconditioningtypeid' & 'architecturalstyletypeid' & 'basementsqft'\\ 
'buildingqualitytypeid'& 'buildingclasstypeid'& 'calculatedbathnbr'\\
'finishedfloor1squarefeet'& 'calculatedfinishedsquarefeet'& 'finishedsquarefeet6'\\
 'finishedsquarefeet15'& 'finishedsquarefeet50'& 'fips'\\
    \hline
  \end{tabular}
\end{center}
 
\section{Approach}
The data consist of a full list of real estate properties in three counties (Los Angeles, Orange and Ventura, California) data in 2016. (https://www.kaggle.com/c/zillow-prize-1/data)
The train data has all the transactions before October 15, 2016, plus some of the transactions after October 15, 2016.

 \end{document}
 
 